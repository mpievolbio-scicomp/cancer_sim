\PassOptionsToPackage{unicode=true}{hyperref} % options for packages loaded elsewhere
\PassOptionsToPackage{hyphens}{url}
%
\documentclass[]{article}
\usepackage{lmodern}
\usepackage{amssymb,amsmath}
\usepackage{ifxetex,ifluatex}
\usepackage{fixltx2e} % provides \textsubscript
\ifnum 0\ifxetex 1\fi\ifluatex 1\fi=0 % if pdftex
  \usepackage[T1]{fontenc}
  \usepackage[utf8]{inputenc}
  \usepackage{textcomp} % provides euro and other symbols
\else % if luatex or xelatex
  \usepackage{unicode-math}
  \defaultfontfeatures{Ligatures=TeX,Scale=MatchLowercase}
\fi
% use upquote if available, for straight quotes in verbatim environments
\IfFileExists{upquote.sty}{\usepackage{upquote}}{}
% use microtype if available
\IfFileExists{microtype.sty}{%
\usepackage[]{microtype}
\UseMicrotypeSet[protrusion]{basicmath} % disable protrusion for tt fonts
}{}
\IfFileExists{parskip.sty}{%
\usepackage{parskip}
}{% else
\setlength{\parindent}{0pt}
\setlength{\parskip}{6pt plus 2pt minus 1pt}
}
\usepackage{hyperref}
\hypersetup{
            pdftitle={CancerSim: A Cancer Simulation Package for python3},
            pdfauthor={Luka Opasic, Jacob Scott, Arne Traulsen, Carsten Fortman-Grotte},
            pdfborder={0 0 0},
            breaklinks=true}
\urlstyle{same}  % don't use monospace font for urls
\setlength{\emergencystretch}{3em}  % prevent overfull lines
\providecommand{\tightlist}{%
  \setlength{\itemsep}{0pt}\setlength{\parskip}{0pt}}
\setcounter{secnumdepth}{0}
% Redefines (sub)paragraphs to behave more like sections
\ifx\paragraph\undefined\else
\let\oldparagraph\paragraph
\renewcommand{\paragraph}[1]{\oldparagraph{#1}\mbox{}}
\fi
\ifx\subparagraph\undefined\else
\let\oldsubparagraph\subparagraph
\renewcommand{\subparagraph}[1]{\oldsubparagraph{#1}\mbox{}}
\fi

% set default figure placement to htbp
\makeatletter
\def\fps@figure{htbp}
\makeatother


\title{CancerSim: A Cancer Simulation Package for python3}
\author{Luka Opasic, Jacob Scott, Arne Traulsen, Carsten Fortman-Grotte}
\date{}

\begin{document}
\maketitle

\hypertarget{background}{%
\subsection{Background}\label{background}}

Cancer is a group of complex diseases characterized by excessive cell
proliferation, invasion, and destruction of the surrounding tissue
~{[}\protect\hyperlink{ref-kumar:book:2017}{1}{]}. Its high division and
mutation rates lead to excessive intratumour genetic heterogeneity which
makes cancer highly adaptable to environmental pressures such as therapy
~{[}\protect\hyperlink{ref-turajlic:NRG:2019}{2}{]}. Throughout most of
its existence tumour is inaccessible to direct observation and
experimental evaluation. Therefore, computational modelling can be
useful to study many aspects of cancer. Some examples where theoretical
models can be of great use include early carcinogenesis, as lesions are
clinically observable when they already contain millions of cells,
seeding of metastases, and cancer cell dormancy
~{[}\protect\hyperlink{ref-altrock:NatRevCancer:2015}{3}{]}.

Here, we present CancerSim, a software that simulates somatic evolution
of tumours. The software produces virtual spatial tumours with variable
extent of intratumour genetic heterogeneity and realistic mutational
profiles. Simulated tumours can be subjected to multi-region sampling to
obtain mutation profiles that are realistic representation of the
sequencing data. This makes the software useful for studying various
sampling strategies in clinical cancer diagnostics. An early version of
this cancer evolution model was used to simulate tumours subjected to
sampling for classification of mutations based on their abundance
~{[}\protect\hyperlink{ref-opasic:BMCCancer:2019}{4}{]}. Target users
are scientists working in the field of mathematical oncology and
students with interest in studying somatic evolution of cancer.

Our model is abstract, not specific to any neoplasm type and does not
consider a variety of biological features commonly found in neoplasm
such as vasculature, immune contexture, availability of nutrients, and
architecture of the tumour surroundings. It resembles the most to
superficially spreading tumours like carcinoma in situ, skin cancers, or
gastric cancers, but it can be used to model any tumour on this abstract
level.

The tumour is simulated using a two-dimensional, on-lattice, agent-based
model. The tumour lattice structure is established by a sparse matrix
whose non-zero elements correspond to the individual cells. Each cell is
surrounded by eight neighbouring cells (Moore neighbourhood). The value
of the matrix element is an index pointing to the last mutation cell
acquired in the list of mutations which is updated in each simulation
step.

The simulation advances in discrete time-steps. In each simulation step,
every tumour cell in the tumour that has an unoccupied neighbour can
divide with a certain probability (params.div\_\_probability). The
daughter cell resulting from a cell division inherits all mutations from
the parent cell and acquires a new mutation with a given probability
(params.mut\_prob). Different division probabilities can be introduced
for some cells in order to simulate variability in fitness of cells that
acquired a beneficial or deleterious mutation. The simulation allows the
acquisition of more than one mutational event per cell
(params.mut\_per\_division). In that case, variable amounts of
sequencing noise ~{[}\protect\hyperlink{ref-williams:NG:2016}{5}{]} can
be added to make the output data more biologically realistic.

Throughout the cancer growth phase, CancerSim stores information about
the parent cell and a designation of newly acquired mutations for every
cell. Complete mutational profiles of cells are reconstructed a
posteriori based on the stored lineage information.

The division rules which allow only cells with empty neighbouring nodes
to divide, cause exclusively peripheral growth and complete absence of
dynamics in the tumour centre. To allow for variable degree of growth
inside the tumour, we introduced a death process. At every time step,
after all cells attempt their division, a number of random cells die and
yield their position to host a new cancer cell in a subsequent time
step.

After the simulation, the tumour matrix, and the lists of lineages and
frequencies of each mutation in the tumour are exported to files.
Furthermore, the virtual tumour can be sampled and a histogram over the
frequency of mutations will be visualised. Alternatively, a saved tumour
can be loaded from file and then subjected to the sampling process.

\hypertarget{installation}{%
\subsection{Installation}\label{installation}}

CancerSim is written in Python (version \textgreater{}3.5). We recommend
to install it directly from the source code. To download the code:

\textbf{EITHER} clone the repository:

\begin{verbatim}
$> git clone https://github.com/mpievolbio-scicomp/cancer_sim.git
\end{verbatim}

\textbf{OR} download the source code archive:

\begin{verbatim}
$> wget https://github.com/mpievolbio-scicomp/cancer_sim/archive/master.zip
$> unzip master.zip
$> mv cancer_sim-master cancer_sim
\end{verbatim}

Change into the source code directory

\begin{verbatim}
$> cd cancer_sim
\end{verbatim}

We provide for two alternatives to install the software after it was
downloaded:

\hypertarget{alternative-1-conda}{%
\subsubsection{Alternative 1: Conda}\label{alternative-1-conda}}

\hypertarget{new-conda-environment}{%
\paragraph{New conda environment}\label{new-conda-environment}}

We provide an \texttt{environment.yml} to be consumed by \texttt{conda}.
To create a fully self-contained conda environment (named
\texttt{casim}):

\begin{verbatim}
$> conda env create -n casim --file environment.yml
\end{verbatim}

This will also install the cancer simulation code into the new
environment.

To activate the new conda environment:

\begin{verbatim}
$> source activate casim
\end{verbatim}

or

\begin{verbatim}
$> conda activate casim
\end{verbatim}

if you have set up conda appropriately.

\hypertarget{install-into-existing-and-activated-conda-environment}{%
\paragraph{Install into existing and activated conda
environment}\label{install-into-existing-and-activated-conda-environment}}

To install the software into an already existing environment:

\begin{verbatim}
$> conda activate <name_of_existing_conda_environment>
$> conda env update --file environment.yml
\end{verbatim}

\hypertarget{alternative-2-using-pip}{%
\subsubsection{Alternative 2: Using pip}\label{alternative-2-using-pip}}

The file \texttt{requirements.txt} is meant to be consumed by
\texttt{pip}:

\begin{verbatim}
$> pip install -r requirements.txt [--user]
\end{verbatim}

The option \texttt{-\/-user} is needed to install without admin
privileges.

\hypertarget{testing}{%
\subsection{Testing}\label{testing}}

Although not strictly required, we recommend to run the test suite after
installation. Simply execute the \texttt{run\_tests.sh} shell script:

\begin{verbatim}
$> ./run_tests.sh
\end{verbatim}

This will generate a test log named
\texttt{casim\_test@\textless{}timestamp\textgreater{}.log} with
\texttt{\textless{}timestamp\textgreater{}} being the date and time when
the test was run. You should see an \texttt{OK} at the bottom of the
log. If instead errors or failures are reported, something is wrong with
the installation or the code itself. Feel free to open a github issue at
\url{https://github.com/mpievolbio-scicomp/cancer_sim/issues} and attach
the test log plus any information that may be useful to reproduce the
error (version hash, computer hardware, operating system, python
version, a dump of \texttt{conda\ env\ export} if applicable, \ldots{}).

The test suite is automatically run after each commit to the code base.
Results are published on
\href{https://travis-ci.org/mpievolbio-scicomp/cancer_sim}{travis-ci.org}.

\hypertarget{highlevel-functionality}{%
\subsection{High--level functionality}\label{highlevel-functionality}}

The parameters of the cancer simulation are given via a python module or
programmatically via the \texttt{CancerSimulationParameters} class. A
documented example \texttt{params.py} is included in the source code
(under \texttt{test/params.py}) and reproduced here:

\begin{verbatim}
$> cat test/params.py
# Number of mesh points in each dimension
matrix_size                      = 100

# Number of generations to simulate.
num_of_generations              = 20

# Number of divisions per generation.
div_probability                 = 1

# Number of division for cells with mutation.
fittnes_advantage_div_prob      = 1

# Fraction of cells that die per generation.
dying_fraction                   = 0.1

# Fraction of cells with mutation that die per generation.
fitness_advantage_death_prob    = 0.0

# Rate of mutations.
mut_prob                        = 1

# Mutation probability for the adv. cells.
advantageous_mut_prob           = 1

# Number of mutations per cell division.
mut_per_division                = 10

# Time after which adv. mutations occur.
time_of_adv_mut                 = 2

# Number of mutations present in first cancer cell.
num_of_clonal                   = 15

# Tumour multiplicity.
tumour_multiplicity             = None

# Sequencing read depth.
read_depth                      = 100

# Fraction of cells to be sampled.
# sampling_fraction             = 0.1
\end{verbatim}

The simulation is started from the command line. The syntax is

\begin{verbatim}
$> python -m casim.casim [-h] [-o DIR] seed
\end{verbatim}

The mandatory command line argument \texttt{seed} is the random seed.
Using the same seed on two simulation runs with identical parameters
results in identical results, this may be used for testing and
debugging. The optional argument \texttt{DIR} specifies the directory
where to store the simulation log and output data. If not given, output
will be stored in the directory \texttt{casim\_out} in the current
directory. For each seed, a subdirectory \texttt{cancer\_SEED} will be
created. If that subdirectory already exists because an earlier run used
the same seed, the run will abort. This is a safety catch to avoid
overwriting data from previous runs.

\hypertarget{example-1}{%
\subsubsection{Example 1}\label{example-1}}

\begin{verbatim}
$> python -m casim.casim 1
\end{verbatim}

\hypertarget{example-2}{%
\subsubsection{Example 2}\label{example-2}}

\begin{verbatim}
$> mkdir sim_out
$> python -m casim.casim -o sim_out 2
\end{verbatim}

Results will be stored in the newly created directory
\texttt{sim\_out/}.

\hypertarget{reference-manual}{%
\subsection{Reference Manual}\label{reference-manual}}

The API reference manual is available at
\url{https://cancer-sim.readthedocs.io}.

\hypertarget{examples}{%
\subsection{Examples}\label{examples}}

See our quickstart example in
\texttt{docs/source/include/notebooks/quickstart\_example.ipynb}.

\hypertarget{bibliography}{%
\section*{References}\label{bibliography}}
\addcontentsline{toc}{section}{References}

\hypertarget{refs}{}
\leavevmode\hypertarget{ref-kumar:book:2017}{}%
{[}1{]} J. C. A. Vinay Kumar Abul K. Abbas, \emph{Robbins Basic
Pathology}, 10th ed. (Elsevier, 2017).

\leavevmode\hypertarget{ref-turajlic:NRG:2019}{}%
{[}2{]} S. Turajlic, A. Sottoriva, T. Graham, and C. Swanton, Nat Rev
Genet (2019).

\leavevmode\hypertarget{ref-altrock:NatRevCancer:2015}{}%
{[}3{]} P. M. Altrock, L. L. Liu, and F. Michor, Nat Rev Cancer
\textbf{15}, 730 (2015).

\leavevmode\hypertarget{ref-opasic:BMCCancer:2019}{}%
{[}4{]} L. Opasic, D. Zhou, B. Werner, D. Dingli, and A. Traulsen, BMC
Cancer \textbf{19}, 403 (2019).

\leavevmode\hypertarget{ref-williams:NG:2016}{}%
{[}5{]} M. J. Williams, B. Werner, C. P. Barnes, T. A. Graham, and A.
Sottoriva, Nature Genetics \textbf{48}, 238 (2016).

\end{document}
