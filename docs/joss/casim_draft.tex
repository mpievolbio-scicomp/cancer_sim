\documentclass[11pt,a4paper]{article}
\usepackage[utf8]{inputenc}
\usepackage{helvet}


\usepackage{authblk}
\usepackage{lineno}
\usepackage{caption}
\usepackage{cite}
\usepackage{url}
\usepackage[utf8]{inputenc}
\usepackage{helvet}
\usepackage{todonotes}
\usepackage{amsmath}
%\usepackage{amthrsfs}
\usepackage{dsfont}
\usepackage{mathrsfs}

\usepackage{amsfonts}
\usepackage{hyperref}
\usepackage{amsthm}
\usepackage{graphicx}
\usepackage{subfigure}
\usepackage{xcolor}
\usepackage{authblk}
%\renewcommand{\qed}{\hfill\small{$\square$}\normalsize}
%alsize}


\DeclareFixedFont{\Acknowledgment}{OT1}{cmr}{bx}{n}{14pt}
\textwidth 150mm \textheight 200mm \hoffset -1.2cm \voffset -0.5cm
\linespread{1.1}


\begin{document}
\title{Python Interactive Cancer Knowledge Analyser}

\author[1]{Luka Opasic}
\author[1]{Carsten Fortmann-Grotte}

\affil[1]{Max Planck Instutute for Evolutionary Biology, Ploen, Germany}

\date{}
\maketitle


\section*{Summary}

Cancer is group of complex diseases characterized by excessive cell proliferation, invasion and destruction of the surrounding tissue. 
Its high division and mutation rates lead to excessive intratumour heterogeneity which makes cancer highly adaptable to environmental pressures such as therapy. 
We have built a cancer model that produces tumours with variable levels of intratumour heterogeneity. 
It is able simulate cancer in multiple spatial dimensions:
(i) Well-mixed cancer which corresponds to haematological neoplasms, 
(ii) Two dimensional on-lattice simulations suitable for modeling superficially spreading tumours like carcinoma in situ and finally
(iii) three dimensional on-lattice simulations for spatial solid tumours. 
It contains two specific division mechanism both ran in discrete time-steps. In the first scenario each time-step only one cancer cell is chosen for division and in the second scenario every cancer cell has a certain probability to divide during one time-step. 
In former case tumour growth is linear, in latter it is exponential if every cell divides each time-step. 
Different division probabilities can be introduced for some cells in order to simulate variability in fitness of cells who acquired beneficial or deleterious mutation.   

During each division mother cell can give birth to one daughter cell and remain unaltered. 
It can also give birth to two daughter cells, one placed in available neighbour location and one daughter cell which replaces the mother cell at the same location.


Our model is abstract and does not consider variety of biological features commonly found in neoplasm such as vasculature, immune contexture, availability of nutrients and architecture of the tumour surroundings. 


Simulated tumours can be pickled via dill package \cite{mckerns:arXiv:2012} and further subjected to virtual biopsy sampling with frequencies of mutations present within the each sample as an output. 
To make output results more biologically sound, recovered frequencies can be passed through function designed to simulate variable sequencing coverage depth and to introduce sequencing noise into the data. 




Simulation is written in Python and can be imported as Anaconda package. 

We implemented growth visualizer for two-dimensional tumour.







Guidance from the journal:
Authors include in the paper some sentences that would explain the software functionality and domain of use to a non-specialist reader. Your submission should probably be somewhere between 250-1000 words. A summary describing the high-level functionality and purpose of the software for a diverse, non-specialist audience.
A clear statement of need that illustrates the purpose of the software. Mentions (if applicable) of any ongoing research projects using the software or recent scholarly publications enabled by it



%\bibliographystyle{humannat}
%\bibliographystyle{vancouver}
\bibliography{\string~/Bibtex/et.bib}



\end{document}
